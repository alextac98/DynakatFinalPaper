\section{PROJECT IMPLEMENTATION}
\subsection{CAD Model Design} \label{sec:CAD Model Design}
To simplify development of the project, we developed a very simple model of a quadruped in Dassault SolidWorks, as seen in figure \ref{fig:cadmodel}. This model has the 4-DoF legs we want, and allows us to simplify the computation needed to more efficiently develop a parametric model of a quadruped.

\begin{figure}[thpb]
    \parbox{\linewidth}{\includegraphics[width=\linewidth]{Figures/robot.png}}
    \caption{The CAD model of the simplified robot.}
    \label{fig:cadmodel}
\end{figure}

\subsubsection*{4-DoF Planar Leg Design} \label{sec:legdesign}
Most companies and research institutions will look toward a 3-DoF leg when designing a quadrupedal robot, and with good reason. Adding unnecessary links and joints make the robot more prone to failure, while complicating dynamic modeling and increasing computational power required. However, we propose that a 4-DoF planar leg, as depicted in Figure \ref{fig:dhframes}, is a good option for off-road quadrupedal robots. The redundant last link gives the engineer the capability to set the contact angle to the ground, potentially increasing performance in rough terrain. This design is also capable of emulating a generic 3-DoF quadruped, due to its similar design. Therefore we believed that this leg design is best for research purposes.

\begin{figure}[thpb]
    \parbox{\linewidth}{\includegraphics[width=\linewidth]{Figures/dhframes.png}}
    \caption{A depiction of the chosen DH frames}
    \label{fig:dhframes}
\end{figure}

\subsubsection*{Importing the Model into Simscape}
We then ported the model into MATLAB using exporting software offered by SolidWorks and we used the Simscape Multibody Simulink library in MATLAB to base the simulation of the robot. For the purpose of simplicity, all legs are equivalent, but given different poses to make it look more like a cat.

\subsection{Collision in Simulink 3D Model}
We used the Contact Forces Library to implement collision between parts of the robot and a plane we had placed underneath the robot. we used primitives called Sphere-to-plane collision blocks, which allow us to model a sphere to plane collision in Simulink. However, without any control inputs on the legs we can't keep the legs straight to demonstrate a standing behavior of the quadruped. We did achieve a dangling pose and a crashing one for the robot, though.

Aiming to fully constraining the model (adding self-collision and body-plane collision) would send us off on a path that may derail us from reaching our reach goal. Figure \ref{fig:collision} shows an older model colliding with the floor.
\begin{figure}[thpb]
    \centering
    \includegraphics[width=\linewidth]{Figures/ContactModel.png}
    \caption{The cat crashing into the floor after falling from a height, proving collision modeling is working between paws and ground}
    \label{fig:collision}
\end{figure}

\subsection{DH-Parameters and Validation} \label{sec:dhparams}
We decided to follow the Denavit-Hartenberg convention when modeling DynaKat, since it is the chosen standard in the robotics industry. Since all of our legs are the same, we specify one set of DH parameters, seen in Table \ref{tab:dhparams}. Figure \ref{fig:dhframes} also shows a visual representation of the DH parameters we chose. The rationale behind choosing the robot's frame $F_w$ having its z axis pointing upward is to make the dynamic model more intuitive, both during development and for future application, given the assumption that gravity is considered to be pointing downward.

\begin{table}
    \begin{center}
        \begin{tabular}{ | c | c | c | c | c |}
            \hline
            link & $\theta_i$       & $d_i$ & $a_i$ & $\alpha_i$       \\
            \hline
            Rot1 & 0                & 0     & 0     & $\frac{\pi}{2}$  \\
            \hline
            Rot2 & $-\frac{\pi}{2}$ & 0     & 0     & 0                \\
            \hline
            1    & $\theta_1$       & 30    & 0     & $\frac{\pi}{2}$  \\
            \hline
            2    & $\theta_2$       & 0     & 100   & $-\frac{\pi}{2}$ \\
            \hline
            3    & $\theta_3$       & 0     & 75    & 0                \\
            \hline
            4    & $\theta_4$       & 0     & 50    & 0                \\
            \hline
            5    & $\frac{\pi}{2}$  & 0     & 0     & $\frac{\pi}{2}$  \\
            \hline
        \end{tabular}
    \end{center}
    \caption{DH-Parameters}
    \label{tab:dhparams}
\end{table}

As stated previously, Table \ref{tab:dhparams} displays the DH parameters based on the robot's world frame. The first two rows correspond to the transformation from the world frame to the first joint's frame, while the next four are those that propagate the frame orientations based on the values of the joint angles. The last row orients the approach vector along the final link's axis.

Figure \ref{fig:dhvalidation} shows the validation of our DH parameters. We used the Robotics Toolbox to ensure our DH parameters were accurate to how we wanted the legs to be described

\begin{figure}[thpb]
    \parbox{\linewidth}{\includegraphics[width=\linewidth]{Figures/dhvalidation.png}}
    \caption{The plot that validates our DH-parameters. The configuration $\theta$ was set to [$\frac{\pi}{2}$ 0 -$\frac{\pi}{2}$ -$\frac{\pi}{2}$] to give the expected configuration in a plot. The Robotics Toolbox for MATLAB was used to generate this plot.}
    \label{fig:dhvalidation}
\end{figure}

\subsection{Forward Kinematics}
Given those DH-parameters shown in \ref{sec:dhparams}, it is almost trivial to compute the forward kinematics. We developed a MATLAB function to return the transformation matrix from each leg's base frame to its end effector. This function is designed to run once, and outputs a symbolic matrix with all variables with DH parameters applied, and symbols replacing the $q$ variables. Therefore, we can reduce computation speed when calculating forward kinematics while running. The matrix also aids in Jacobian computation, reducing overhead.

\subsection{Inverse Kinematics}
There are several methods to create an inverse kinematic model of a manipulator, including analytical and heuristic methods. At first, we tried calculating the inverse kinematics using a geometric approach. However, throughout our research, we discovered many other methods to calculating the inverse kinematics of a manipulator. As talked about in Section \ref{sec:Dynamics}, we calculated the inverse velocity kinematics, and represented a trajectory in task-space. Finally, we interpolated the intermediate joint angles to get our final positions. This method was used in the dynamic model demonstration.

We also looked at using a heuristic method of computing our inverse kinematics. We chose FABRIK, since it is fast, resistant to singularities, and capable of modeling multiple end effectors. More about this algorithm is talked about in \ref{sec:fabrik}. In the end, we created a demonstration showing the FABRIK algorithm outputting joint angles for a leg going in a pre-defined trajectory. 

\subsection{The Jacobian}
The Jacobian of an individual leg is the single most important aspect in the process of dynamically modeling the quadruped in question. In fact, we need to be able to compute more than one Jacobian for each configuration, one for each link in the manipulator/leg.
For this purpose, we referred to the book "Robot Modeling and Control" \cite{spong2006robot}, which gives a cookbook method of computing the Jacobian for a point along one of the links of an n-link manipulator. Based on that, we created a function that takes the index of the joint and returns a corresponding 6x4 Jacobian matrix. For the purpose of dynamically modeling the robot, we do not need the matrix to be square. However, for the purpose of task-space trajectory tracking we need the pseudo-inverse.

\subsection{The Dynamics} \label{sec:Dynamics}
In order to control the robot using torque, we need to provide a signal that takes into account the dynamics of the system, hence the need for a dynamical model. To acquire one, we followed the cookbook method shown in the "Robot Modeling and Control" \cite{spong2006robot} and \cite{murray2017mathematical}. We used the first resource to get the general equations and the second to resolve some ambiguity about whether the \textbf{Inertia Tensors} used in the equations were defined with respect to the center of the link or its hinge point. Both resources are complementary and discuss examples that highlight different nuances.
From \cite{spong2006robot}, we got the general equation of the Inertia Matrix:
$$
    M(q) = \sum_{i=1}^n m_i J_{vi}(q)^T J_{vi}(q) +J_{\omega i}(q)^T R_i(q) I_i R_i(q)^T J_{\omega i}(q)
$$
where i is in the index of the center of mass of the $i^{th}$ link, $J_{vi}$ the linear velocity part of the Jacobian, and  $J_{\omega i}$ the rotational part of it for the center of mass of the $i^{th}$ link. In the rotational part of the equation, $I_i$ is the constant inertia tensor defined with respect to the $i^{th}$ link, and $R_i$ the rotation matrix that aligns the inertia tensor to the link with respect to the base-frame. $R_i(q) I_i R_i(q)^T$ is thus the inertia tensor of the $i^{th}$ link with respect to the base frame, which is a function of q.
We acquired the inertia tensors from the SolidWorks model of the leg that we had designed and the rotation was computed by combining the rotation that transforms coordinates from the base to the end of the $i^{th}$ link and another which we designed by hand so that the inertia tensor is aligned with the one represented in SolidWorks. The process was iterative, but we could not validate whether we had the correct configuration until we were able to test it on the model itself.
The other terms are easy to acquire from the properties of the robot, as well as the Jacobian computations.
Then, the elements of the Coriolis and Centrifugal forces matrix were computed by using M's elements and the following equation:
$$
    c_{kj} = \sum_{i=1}^n \{\frac{\partial d_{kj}}{\partial q_i} + \frac{\partial d_{ki}}{\partial q_j} - \frac{\partial d_{ij}}{\partial q_k} \}\dot{q_i}
$$
where $c_{kj}$ is the intersection of M's $k^{th}$ row and M's $j^{th}$ column.
Then, to get the symbolic G vector, we get the system's potential energy as follows:
$$
    P = \sum_{i=1}^n m_i g^T r_{c_i}
$$
where $g^T r_{c_i}$ is the project of the distance to the base-frame of the $i^{th}$ link. We leave the variable g parametric, that is, we let be a symbolic vector and based on the situation in which the arm/leg find itself in, we would then set this g accordingly to counter the effect of gravity on the arm.
Differentiating this equation with respect to the vector of joint values, we obtain the G vector, which is the gravity vector of a manipulator. The following shows the equation we referred to:
$$
    g_k = \frac{\partial P}{\partial q_k}
$$

The resulting matrices can be used to express the dynamic model of the robot which goes like this:
$$
    M(q) \ddot{q} + C(q, \dot{q}) \dot{q} + G = \tau
$$

Implementation-wise, the M, C and G matrices were computed in a file called generatingDynamics.m, which initializes the parameters like weights and inertia tensors and then calls another function which returns the Jacobians for the chosen center of mass, in addition to the rotation that aligns the corresponding inertia matrix and $r_{c_i}$ vectors (the vector that indicates the location of the center of mass of the $i^{th}$ link with respect to the base). Once the matrices are computed, they are turned from symbolic matrices, which are very slow to evaluate, into matlab functions using the matlabFunction routine which is part of the symbolic toolbox. The result is a set of functions, computeM, computeC and computeG which combined run at an average of 25 milliseconds, which is good enough for simulation. Figure \ref{fig:dhframes} also shows the frame in which the dynamic model is shown. Note that in this frame, the gravity can sometimes be in another orientation than the negative $z_world$ vector.

Since we didn't have control over how SimuLink loads the models, we had to do some calibration when designing the controllers that use the model internally. For example, the measured q and \dot{q} sometimes had to be negated. Also, the q values needed some calibration so that the home configuration is as shown in figure \ref{fig:dhframes}. Figure \ref{fig:calibration} shows the setup that was explained.

\begin{figure}[thpb]
    \parbox{\linewidth}{\includegraphics[width=\linewidth]{Figures/controller_calibration.png}}
    \caption{A blown-up SimuLink schematic of a controller for a leg.}
    \label{fig:calibration}
\end{figure}